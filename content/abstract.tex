% !TEX root = ../dkilleffer-thesis-proposal.tex
%
\pdfbookmark[0]{Abstract}{Abstract}
\chapter*{Abstract}
\label{sec:abstract}
\vspace*{-10mm}

%\blindtext

%\vspace*{20mm}

%% removed the alternative "Abstract (cufferent language) section"
%{\usekomafont{chapter}Abstract (different language)}\label{sec:abstract-diff} \\

%\blindtext

%Video is produced at a massive rate each day since more and more people are carrying around very powerful cameras and video recording devices in their pockets...  YouTube.com, one of the most popular websites for user-generated video content, last reported that it was routinely receiving an average of [XXX] gigabytes of video every minute.

There are very few good tools available that allow people to digitally categorize recorded videos.  This is a problem because with the rise of digitization of recordings previously created on magnetic media and other media as well as the profileration of smartphones, there is an ever-growing volume of videos being made available to people, but a severe lack of ways to organize and search those videos.  With the rise of cheap, large online storage resources and very powerful handheld recording devices, but few and inadequate tools to catalog them, there is a looming problem that many videos will go unwatched and uncared for, and the importance of their content go forgotten. By utilizing an easy to use web interface, this project attempts to bring the power of crowds to aid in adding metadata and meaningful annotations to online video for the benefit of other viewers. 



%People are recording more videos today than at any other time in history. Most people have very well equipped video recording devices that they carry in their pockets (smartphones) which have capabilities that dwarf even the most advanced handheld camcorders of just a few years ago. YouTube reports that they receive 300 hours of recorded video submissions per minute (see: https://www.youtube.com/yt/press/statistics.html, accessed 2015-05-22, http://www.tubefilter.com/2014/12/01/youtube-300-hours-video-per-minute/, accessed 2016-02-25, and http://expandedramblings.com/index.php/youtube-statistics/, accessed 2015-02-25).  In addition to the stratospheric proliferation of newly recorded digital video, in recent years an entirely new industry of video preservation companies has sprung up which offer a variety of services to help both professionals and consumers to digitize their recordings made on film, magnetic, or optical formats; these services will accept all manner of defunct formats and digitize the recordings captured on them in as high resolution as possible and create digital copies of the originals.  With the future viability of nearly all physical media formats in doubt at best and al but assuredly over at worst, it seems the writing is on the wall for the future primacy of digital video as the format of the future.  However, with the massive increase in the amount of recorded video content created, how should content be organized for sharing and annotated for posterity?  How can otherwise random, disorganized, disconnected videos be arranged and grouped together to create a cohesive narrative?  This project will present a new platform to allow groups of individuals to collaboratively build up a rich set of metadata around a set of videos that would otherwise be very difficult to scan through and label.  The platform will allow for users to tag individuals that appear in videos so that others can easily search and find when certain people appear in videos, allow for various other forms of tagging and annotation, and also present a rich, faceted video search interface for groups of videos.  The platform will help bring new life and utility to recordings which have long laid dormant.  By utilizing the power of the crowd, rich metadata can be added to videos and make them more interesting to others, and make it much simpler to find that old vacation video from the 1980's, or that funny moment at an old family barbecue.


