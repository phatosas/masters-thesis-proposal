% !TEX root = ../dkilleffer-thesis-proposal.tex
%
\chapter{Design}
\label{sec:design}


\section{Overview}
\label{sec:design:overview}

The prototype will take the form of a website, and will be created using \hyperref[glossary:node.js]{Node.js}, \hyperref[glossary:express.js]{Express.js}, HTML, CSS, and JavaScript for the backend.  The frontend will make use of \hyperref[glossary:bootstrap]{Bootstrap} for styling and responsive layout, as well as \hyperref[glossary:react.js]{React.js} for dynamic user interface management, particularly for managing video annotations.  Additional libraries may be used to aid with the video player display and management.  My selection of an almost exclusively JavaScript based technology stack is based on my desire to gain further experience with JavaScript as a server-side technology, as well as to minimize the "context-switching" costs for myself that I would otherwise have to undergo if using a more mixed-mode technology stack (for more background on context switching and the costs involved, see: \url{http://www.newvoicemedia.com/blog/agile-development-context-switching-comes-at-the-price-of-delivery/}; \url{https://psygrammer.com/2011/03/28/unnecessary-context-switches-the-myth-of-multitasking/}; \url{http://www.petrikainulainen.net/software-development/processes/the-cost-of-context-switching/}; \url{http://www.joelonsoftware.com/articles/fog0000000022.html}; \url{http://www.bryanbraun.com/2012/06/25/multitasking-and-context-switching}; \url{http://blog.ninlabs.com/2013/01/programmer-interrupted/}).


\cite{WEB:NewVoiceMedia:2012,WEB:Psygrammer:2011,WEB:Kainulainen:2014,WEB:Spolsky:2001,WEB:Braun:2012,WEB:Ninlabs:2013}


\section{Technologies Used}
\label{sec:overview:technologies-used}

In contrast to some earlier work, a key goal of this prototype is for the tool to be a purely online system, allowing for web-based collaboration between users.  Another core requirement of the system is that a faceted search be supported to allow for easy discovery of videos of interest to users.  To support the faceted searching of videos, \hyperref[glossary:elasticsearch]{Elasticsearch} will be used as a document repository of all video annotation data.  When users add annotations to a video, that metadata will be added to Elasticsearch and made available for searches on videos, as well as be used to display top categories of videos.

Annotations will be stored in a document-based repository, \hyperref[glossary:elasticsearch]{Elasticsearch}.  Document based repositories such as \hyperref[glossary:apache-solr]{Apache Solr} and \hyperref[glossary:elasticsearch]{Elasticsearch} are commonly used to power search engines because of their capabilities such as fast search indexing, flexible document schema support, and rich support of a wide variety of client libraries.  Interactions with the Elasticsearch document respository are executed via a \hyperref[glossary:restful-api]{RESTful API} and \hyperref[glossary:json]{JSON} messages are exchanged to read/write documents to the repository.

%Due to the loosely-structured nature of the documents that are stored in the Elasticsearch repository and the free-form nature of user-generated video annotations, 

