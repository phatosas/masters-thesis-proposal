% !TEX root = ../dkilleffer-thesis-proposal.tex
%
\chapter{Motivation and Problem Statement}
\label{sec:problem-statement}

%\cleanchapterquote{You can’t do better design with a computer, but you can speed up your work enormously.}{Wim Crouwel}{(Graphic designer and typographer)}
%
%\Blindtext[2][2]
%
%
%\section{Postcards: My Address}
%label{sec:intro:address}
%
%\textbf{Ricardo Langner} \\
%Alfred-Schrapel-Str. 7 \\
%01307 Dresden \\
%Germany
%

%\section{Motivation and Problem Statement}
\section{Motivation}
%\label{sec:problem-statement:motivation}
\label{sec:problem-statement:motivation}

%\Blindtext[3][1] \cite{Jurgens:2000,Jurgens:1995,Miede:2011,Kohm:2011,Apple:keynote:2010,Apple:numbers:2010,Apple:pages:2010}

%\section{Results}
%\label{sec:intro:results}
%
%\Blindtext[1][2]


Today people are recording more videos than at any other time in history. Most people have very well equipped video recording devices that they carry in their pockets (smartphones) which have capabilities that dwarf even the most advanced handheld camcorders of just a few years ago.  YouTube reports that they receive 300 hours of recorded video submissions PER MINUTE (see: \url{https://www.youtube.com/yt/press/statistics.html}, accessed 2015-05-22, \url{http://www.tubefilter.com/2014/12/01/youtube-300-hours-video-per-minute/}, accessed 2016-02-25, and \url{http://expandedramblings.com/index.php/youtube-statistics/}, accessed 2015-02-25).  In addition to the stratospheric proliferation of newly recorded digital video, in recent years an entirely new industry of video preservation companies has sprung up which offer a variety of services to help both professional and consumer customers to digitize their recordings made on film, magnetic, or optical formats; these services will accept all manner of defunct formats and digitize the recordings captured on them in as high resolution as possible and create digital copies of the originals.  With the future viability of nearly all physical media formats in doubt at best and al but assuredly over at worst, it seems the writing is on the wall for the future primacy of digital video as the format of the future.  However, with the massive increase in the amount of recorded video content created, how should content be organized for sharing and annotated for posterity?

One popular method people used to use for creating "playlists", "favorite" recordings, and generally sharing of media in the past was to create so-called "mix-tapes" of favorite music and videos.  People would typically use two "decks" to create the compilations; one would be used to play back the song or video they wanted to record, an the other deck would be used to create the new recording.  Today this practice is an all-but forgotten media artifact of the 1980s and 1990s (rarely ever seen since the early 2000s).  At the same time as people are no longer creating and sharing "mix-tapes" of either songs or videos for friends and family, the world is seeing the largest exponential growth of recorded media in its history; clearly the manual labor and time investment required in the now lost art of creating "mix-tapes" does not scale, and better solutions are needed for organizing, categorizing, and sharing important digital video works.  Without the benefit of carefully crafted and curated metadata to catalog the output of this new explosion of recordings, the future usefulness and viability of new recordings is in serious question.  As video recording has shifted away from being created on physical media formats to digital formats, new digital recordings do not have the advantage of their physical media forebears which could be easily and simply labeled with a pen or marker to describe their contents.

At the same time as society has seen explosive growth in the proliferation of recordings due to technological advances, we have also seen the rise of the "social" web.  People are sharing all aspects of their lives with friends, family, even perfect strangers online.  Participants in the "social web" allow others to add metadata, comments, and add to their own digitally shared pictures, music, videos, etc.  This has proven to be a very effective way to apply meaningful metadata to digital artifacts, and adds to the future longevity and viability of such digital artifacts (presuming that the metadata applied to individual files can be guaranteed to survive alongside the digital files themselves well into the future).



%\textbf{Chapter \ref{sec:problem-statement:motivation}} \\[0.2em]
%\blindtext

\section{Personal Story}
\label{sec:problem-statement:personal-story}

Several years ago my siblings and I lost our two remaining grandparents; my maternal grandmother, and my paternal grandfather, both within a couple years of each other.  These were difficult losses to take, and I was left thinking of them often and the times we had together.  At the same time, I was recently married and had started a family of my own, and enjoying all the highs, lows, and excitement of being a part of a young family with children.  With sentimentality creeping into my mind more and more, we would often videotape our young son, and then his younger brother, and making memories with our children and saving them to video for posterity.  Several years later I found myself with a very large collection of both VHS and Mini-DV tapes (well over a hundred tapes, at least), and I realized that my wife, kids, and I had not viewed most of these recordings, not even since they were originally taped.  At an extended family dinner I inquired about all the old VHS tapes my parents had recorded of us kids growing up, including several important family milestones and a few select events that my grandparents would have been a part of; I was told that I was free to take any videotapes I could find.  I did the same thing with my in-laws and gathered up all their old videotapes as well; soon I found myself in the posession of a virtual mountain of videotapes, some up to 25 years old.  I knew that over time videotapes degrade and are subject to a process of "vinegarization" (see \url{http://www.clir.org/pubs/reports/pub54/2what_wrong.html} and \url{http://www.emeraldartservices.com/visual-media-deterioration/}), and I knew that if I wanted to preserve all the precious memories that were captured on those tapes that I would need to digitize this collection.

Thus began a process of over a year's worth of work whereby I slowly digitized nearly 200 analogue tapes of various formats; VHS, Mini-DV, VHS-C, etc.  Initially I was tempted to edit and cleanup the recordings as I imported them, but soon I found out just how much time and effort is involved in doing high-quality video editing and cleanup, and I realized I would never finish digitizing the tape collection if I stopped digitizing to edit each tape.  Eventually I was able to successfully digitize about 99\% of the videotape collection, and I was left with an enormous set of video files (one per tape).  Some statistics of the collection:

\begin{itemize}[noitemsep]
%\item TODO: number of video files
\item over 3000 different video files of various formats
%\item TODO: total disk size of all video files
\item over 2.33 TB of disk space used
%\item TODO: total number of hours of recordings
\item over 200 total hours worth of recordings
\end{itemize}

While I certainly loved being able to go back and re-live many funny childhood moments that were now immortalized on an external hard drive, I quickly came to realize several things about my new massive video file collection:

\begin{enumerate}[noitemsep]
\item I did not know \textbf{\textit{who}} was in several of the videos, but I knew that my mom/my dad/my father-in-law/my uncle/etc. would know
\item I did not know \textbf{\textit{when}} many of the recordings took place
\item I did not know \textbf{\textit{where}} some of the recordings were made
\item It was unclear \textbf{\textit{why}} some recordings were made, and not easy to \textbf{\textit{decipher the purpose}} without viewing the video in its entirety (something I did not have the time to do when digitizing the entire old tape collection)
\item I did not know \textbf{\textit{where}} many of the recordings were made
\item It was \textbf{\textit{extremely difficult and time-consuming}} to be able to properly \textbf{\textit{share}} old historical family moments from the video collection with the family because the videos were only labeled by type (VHS, Mini-DV, etc.) and tape number
\end{enumerate}

The last point is perhaps the most poignant. In my eagerness and excitement to share my newfound digital video treasure trove with my family, I fumbled at several family dinners and gatherings when I was requested to play the \textit{"funny barbecue video from 1987 where Nate accidentally hit cousin Al in the head"}, and several other classic family gems; I simply couldn't find videos that I was looking for without basically brute-forcing my way and playing each video, fast-forwarding through the video until either I found (or did not find) what I was looking for, and then wrote down what video and at what time "that moment" that I was looking for was found in.  The problem was that despite having invested over a year's worth of time and effort into preserving all these old family videos, it wasn't worth much to everybody else (or me!) in their current form \textit{\textbf{because nobody could find what they actually wanted to see without watching the entire collection}}.  And this revelation was the genesis of my thesis idea; to create a platform where I could leverage the knowledge and memories of my family to help me build up a set of rich metadata to annotate the family video collection, and then to reward them for their help in annotating the videos by building out a rich, faceted search interface to the video collection, as well as creating the ability to create "playlists" where users can make their own "highlight reels" of special moments, people, places, etc.



\section{Problem Statement}
\label{sec:problem-statement:description}

The problems that my thesis thus attempts to address are:

\begin{itemize}[noitemsep]
\item how to \textbf{\textit{effectively annotate}} and add valuable metadata to a \textbf{\textit{shared collection of video recordings}} via a loosely affiliated group of friends and family
\item how to best \textbf{\textit{organize and present valuable user-added metadata}} from video recordings in \textbf{\textit{a faceted search interface}} that provides an easy way for people to find and play back videos they might otherwise not even know exist
\item enabling users to \textbf{\textit{explore libraries of richly annotated videos}} and \textbf{\textit{make "playlists" of their favorite videos/segments}} based on the metadata that the videos have been tagged with
\end{itemize}

My thesis project will attempt to answer these questions by architecting a web application prototype where users can add videos, annotate who is in those videos, what the content of those videos is, where they take place, when they take place, and more.  This metadata will be aggregated and used to power a faceted search, which will expose the video collection in a powerful new way to users and allow them to explore and reengage with the video collection in new and compelling ways.


%
%
%
%\subsection{Some References}
%\label{sec:problem-statement:results:refs}
%\cite{WEB:GNU:GPL:2010,WEB:Miede:2011}
%
%\section{Thesis Structure}
%\label{sec:problem-statement:structure}
%
%\textbf{Chapter \ref{sec:related}} \\[0.2em]
%\blindtext
%
%\textbf{Chapter \ref{sec:system}} \\[0.2em]
%\blindtext
%
%\textbf{Chapter \ref{sec:concepts}} \\[0.2em]
%\blindtext
%
%\textbf{Chapter \ref{sec:concepts}} \\[0.2em]
%\blindtext
%
%\textbf{Chapter \ref{sec:conclusion}} \\[0.2em]
%\blindtext
%


